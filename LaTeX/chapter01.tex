\section{Need finding}
Si osservano gli utenti nel loro ambiente, mentre svolgono attivit\`a quotidiane, così da invididuarne le necessit\`a, obiettivi e problemi. Pertanto \`e importante osservare:
\begin{itemize}
	\item senza avere chiaro cosa cercare;
	\item senza coinvolgere direttamente l'utente;
	\item senza condizionare l'utente;
	\item imparando i vari passi del processo;
	\item le emozioni/paure/frustrazioni dell'utente.
\end{itemize}
I \textit{need} non sono \textit{soluzioni}. Queste ultime limitano l'innovazione, mentre i \textit{need} aprono nuove possibilit\`a.

\subsection{Tecniche di \textit{need finding}}
\begin{enumerate}
	\item interviste: \\Informali ed economiche, permettono di far scoprire informazioni soggettive inaspettate, ma richiedono parecchio tempo. Vengono fatte a persone rappresentative degli utenti o a utenti di sistemi simili. Si preparano prima le domande, si cerca di registrare l'intervista, si incoraggia l'intervistato ad approfondire le risposte. Evitare domande ovvie, o che contengono la risposta, o quelle troppo generiche; andare pi\`u sul concreto e chiedere il perch\`e di risposte date;
	\item interviste specifiche:
	\begin{itemize}
		\item ai \textit{lead users}: utenti particolarmente competenti e sofisticati, che hanno dei \textit{need} nuovi per i quali sono in grado di trovare soluzioni; sono in grado di prevedere i \textit{need} che molti utenti avranno in un futuro prossimo;
		\item agli \textit{extreme users}: utenti che spingono un sistema all'estremo, trovando problemi altrimenti difficili da individuare;
		\item a esperti: in queste interviste si sfruttano le conoscenze degli esperti, che possono quindi discutere di problemi pi\`u difficili o astratti;
		\item \textit{history interviews}: interviste utili a comprendere sequenze di eventi che spiegano il comportamento degli utenti;
		\item \textit{process mapping}: interviste in cui si chiede all'intervistato di descrivere tutto il processo;
		\item \textit{laddering}: ricorrere al frequente utilizzo della domanda \textit{perch\`e?};
		\item \textit{cultural context}: interviste non strutturate, per conoscere il contesto;
		\item \textit{intercepts}: intervista a domanda singola.
	\end{itemize}
	\item questionari: \\Vengono utilizzate delle domande uguali per tutti e permettono di raccogliere molte informazioni velocemente sulle quali \`e possibile fare un'analisi pi\`u appronfodita. Vengono usati molti tipi di domande (aperte, chiuse, risposta singola, risposta multipla, scala di valori, ranking, \ldots). Sono d'altra parte meno flessibili delle interviste. Si possono utilizzare diversi strumenti online per realizzarli;
	\item diario: utilizzato per raccogliere informazioni su attivit\`a sporadiche, per studi longitudinali;
	\item \textit{pager studies}: si chiede all'intervistato di fermarsi e prendere nota in tempo reale, in un particolare momento/posto, in modo completamente asincrono, utilizzando note libere o form da riempire;
	\item \textit{camera studies}: riprendere l'attivit\`a svolta dall'utente per vederla con i \textit{suoi} occhi;
\end{enumerate}