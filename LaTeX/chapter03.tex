\section{Realizzazione del prototipo}
Per progettare la \textit{UI} occorre immaginare l'interazione dell'utente con l'applicativo ai fini della risoluzione di un \textit{task} specifico, considerando come strumenti disponibili dispositivi hardware, widget e i vari stili di interazione. \\
Per capire se la \textit{UI} progettata \`e completa occorre analizzarne la comprensibilit\`a e la facilit\`a di utilizzo. \\
Il prototipo \`e un modello di un sistema interattivo che simula e/o anima alcuni aspetti/caratteristiche/funzioni dell'applicativo finale, utilizzato per valutare l'impatto sugli utenti e dove le condizioni di valutazione devono essere simili a quelle previste per l'interfaccia finale. \\
I prototipi vengono utilizzati per meglio definire l'applicativo e per testarne una versione iniziale, intermedia o finale. \\
Esistono tre approcci:
\begin{enumerate}
	\item \textit{throw-away}: viene utilizzata tutta la conoscenza appresa del contesto per la realizzazione del prototipo;
	\item \textit{incrementale}: le varie funzioni vengono aggiunte una alla volta al prototipo;
	\item \textit{evolutivo}: un prototipo realizzato viene utilizzato come base per il successivo, che lo migliora.
\end{enumerate}
I problemi principali della prototipazione riguardano i tempi, perch\`e richiede tempo, e se persistono problemi, il tempo impegnato sulla realizzazione pu\`o sembrare buttato. Inoltre, \`e difficile pianificare un processo di design con la prototipazione, \`e difficile valutarne i costi e prescinde completamente dalla sicurezza, l'affidabilit\`a e i tempi di risposta dell'applicativo.

\subsection{Paper prototyping}
Si realizzano i prototipi su carta, molto rapidamente, in modo da essere pronti per essere buttati e rifatti. Si disegnano tutte le componenti di un'eventuale \textit{UI} (\textit{label}, bottoni, \ldots), ma senza dar troppo peso all'estetica, cos\`i da non perder troppo tempo perch\`e l'utente dovr\`a essere interessato e concentrato solo sui task. \\
Si mostra il prototipo di carta ad un potenziale utente, ed un \textit{human computer} si occupa di cambiare i foglietti all'interazione dell'utente con l'interfaccia. \\
I vantaggi riguardano la loro facilit\`a di creazione, la possibilit\`a di fare test subito con utente e che questo non \`e interessato alla grafica ma ai contenuti; permettono di testare la navigazione, il \textit{workflow}, la terminologia e la funzionalit\`a.

\subsection{Ausilio dei computer}
Esistono strumenti che permettono di lavorare in digitale sui \textit{paper prototypes}: si fotografano i disegni e si importano, si marcano le zone cliccabili e vi si associano altri disegni fotografati. In questo modo si sostituisce lo \textit{human computer} e si possono loggare i test.