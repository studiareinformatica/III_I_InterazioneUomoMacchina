\section{Goals, vincoli e trade-offs}
I \textit{goals} sono gli obiettivi che descrivono lo scopo per cui si sta progettando il sistema, per chi lo si sta facendo e il perch\`e eventuali utenti dovrebbero volere il nostro design. \\
I \textit{constraints} (o vincoli) sono i materiali e gli standard utilizzati dalla nostra applicazione, o i costi, o i framework e le piattaforme, o il tempo necessario. \\
I \textit{trade-offs} (o compromessi) sono il costo da dover pagare per poter soddisfare determinati \textit{goal} o \textit{constraint}, sono i \textit{goal} e \textit{vincoli} sacrificati per altri \textit{goal} e \textit{vincoli}.

\section{Design}
\`E di fondamentale importanza prevedere un design per la massima utilizzabilit\`a dell'applicativo, pertanto il lavoro del designer \`e un lavoro di gruppo, che include la raccolta e il coordinamento di un insieme multidisciplinare di compentenze, dove il vero designer non impone il suo stile, ma sceglie materiali e tecniche e sperimenta, tenendo conto della componente psicologica e dei costi. \\
Pertanto, il \textit{designer \`e un progettista dotato di senso estetico che lavora per la comunit\`a} (\textit{B. Munari}, 1971), compie scelte razionali, oggettive, mentre l'artista fa scelte soggettive.

\section{Modello \textit{Waterfall}}
Il modello di sviluppo \textit{a cascata} (o \textit{waterfall}) prevede le seguenti fasi:
\begin{enumerate}
	\item \textit{specifiche tecniche}: si raccolgono le informazioni necessarie ad individuare cosa il sistema dovr\`a essere in grado di offrire;
	\item \textit{design architetturale}: descrizione di alto livello su come il sistema sar\`a in grado di offrire i servizi richiesti;
	\item \textit{design dettagliato}: affinamento delle componenti architetturali e delle interrelazioni per identificare i moduli da realizzare separatamente;
	\item \textit{sviluppo e testing}: costruzione del sistema;
	\item \textit{integrazione e testing} integrazione del sistema;
	\item \textit{manutenzione}.
\end{enumerate}
Spesso questo modello fallisce perch\`e alcune delle fasi variano durante lo sviluppo ma non rappresenta un modello ingrado di tornare indietro facilmente, perch\`e \`e un modello linerare e sequenziale.

\section{Modello Agile}
Molto pi\`u adattivo rispetto al modello a cascata, perch\`e utilizza metodi orientati alle persone, piuttosto che al processo. \`E un modello circolare, iterativo, che prevede \textit{pre-release} e \textit{release} cicliche e rapide, \`e collaborativo con team multidisciplinari e prevede anche l'apporto dell'utente finale. \\
Prevede due \textit{loop} principali:
\begin{itemize}
	\item \textit{user-research}/design
	\item \textit{design/development}
\end{itemize}