\section{Tasks}
Una volta individuati i \textit{need} occorre capire quali di essi la nostra interfaccia si propone di soddisfare, cercando di capire i motivi che spingerebbero un eventuale utente ad utilizzare l'interfaccia e cosa questa gli permetterebbe di fare. \\
Per questo si individuano i \textit{task}: delle sub-attivit\`a che l'utente deve svolgere nel corso dell'attivit\`a generale per arrivare a soddisfare i \textit{need} selezionati. \\
Pertanto ci interessa il ruolo della \textit{UI}, ma non le sue funzioni, gli utenti coinvolti e il contesto.

\subsection{Storyboards}
Per definire i \textit{task} si utilizza disegnarli come \textit{storyboard}, dei semplici fumetti disegnati a mano libera, con poche (da 2 a 4, massimo) scheramte e poco testo. \\
In questi fumetti non si disegnano schermate dell'applicativo, perch\`e ancora non progettate ed altrimenti ci vincolerebbero. \\
I vantaggi di questo sistema sono:
\begin{enumerate}
	\item mostra un sistema e il suo contesto d'uso;
	\item permette di condividerlo con gli altri;
	\item non lega subito ad un'interfaccia particolare;
	\item pu\`o essere disegnato (e corretto) in pochi minuti;
	\item permette la discussione e il miglioramento delle scelte.
\end{enumerate}
Esistono strumenti che permettono di creare gli \textit{storyboard} mantenendo lo stile-fumetto.
\paragraph{In sintesi}
Si disegna uno \textit{storyboard} per ciascun task, concentrandosi sui \textit{task} principali, che in linea di massima devono essere compresi tra 2 e 4.