\section{Luogo dell'attenzione}
Il luogo dell'attenzione è un oggetto fisico o un'idea alla quale stiamo pensando attivamente ed intenzionalmente.
\begin{itemize}
	\item \textit{focus}: volonta di focalizzare l'attenzione su qualcosa, mediante un azione;
	\item \textit{locus}: non si pu\`o controllare completamente, perch\`e l'attenzione viene attratta da qualcosa, che non coincide con il \textit{focus} dell'interfaccia.
\end{itemize}
Quando si \`e \textit{assorti} ci si trova in una situazione in cui abbiamo un solo luogo dell'attenzione, ma un evento pu\`o spostarlo. \\
Quando delle azioni (o compiti) vengono effettuate ripetutamente, creano un'abitudine, che permette di esseguirle senza pensarci. \\
I compiti che eseguiamo senza pensiero cosciente sono detti \textit{automatici}: tutti i compiti che eseguiamo contemporaneamente - tranne uno - sono automatici, quello non automatico riguarda il luogo dell'attenzione. \\
L'\textit{interferenza} \`e il tentativo di eseguire contemporaneamente due compiti non automatici, che ci fa peggiorare il rendimento in entrambi. \\
Le \textit{sequenze di azioni} eseguite ripetutamente diventano dei task automatici, cio\`e un abitudine. Interrompere una sequenza automatica richiede lo spostamento dell'attenzione su di essa. \\\\

Quindi, l'uso sistematico di un'interfaccia crea delle abitudini, e questo pu\`o essere sfruttato dal designer, perch\`e l'utente tende a prendere delle abitudini e l'interfaccia pu\`o aiutarlo a farlo. \\
L'interfaccia dovrebbe permettere la concentrazione sul task, ma se l'esecuzione del task \`e difficile o delicata, l'utente tende a concentrarsi solo su alcuni aspetti essenziali, ignorando warning e help forniti dall'interfaccia. Ma se sappiamo dove si trova il luogo dell'attenzione possiamo apportare cambiamenti in altre parti del sistema, senza distrarre l'utente. \\
Generalmente un cambio di contesto richiede 10 secondi, se per\`o diventa abituale, pu\`o richiedere molto meno. Pertanto, \`e importante che l'applicati, alla terminazione di alcuni task, sia in grado di tornare al contesto precedente.

\section{Modi}
Il \textit{content} (contenuto) \`e l'insieme di informazioni che risiedono in un sistema, utili all'utente. \\
Il \textit{GID} (\textit{graphical input device}) \`e il meccanismo per comunicare al sistema una particolare locazione o la scelta di un oggetto (tipicamente la posizione del cursore). \\
Il \textit{GID button} \`e il bottone principale del \textit{GID}. \\
Il \textit{tap} \`e l'azione di premere e rilasciare un tasto (che torna al suo stato originale). \\
Il \textit{to click} \`e il posizionamentod del \textit{GID} e il successivo \textit{tap} del \textit{GID button}. \\
Il \textit{to drag} \`e la pressione del \textit{GID button} e, senza rilasciarlo, lo spostamento del \textit{GID}, per poi rilasciarlo in altra locazione. \\
Una \textit{gesture} (gesto) \`e una sequenza di azioni umane completata automaticamente una volta avviata (scrivere una parola comune, tipo \textit{che}). \\
Dato un gesto, un'interfaccia \`e in un \textit{modo} se l'interpretazione di quel gesto \`e sempre la stessa. Quando il gesto viene interpretato in maniera diversa, l'interfaccia \`e in un altro \textit{modo}. Per esempio, il \textit{CAPS LOCK} crea un \textit{modo}. \\
I \textit{quasimodes} sono \textit{modi temporanei} (\textit{modo} che svanisce dopo l'uso, come il pennello di \textit{word}), o \textit{quasi-modi} (\textit{modo} ottenuto attivando e mantenendo fisicamente un controllo, come il tasto \textit{CTRL}). \\
I \textit{noun-verb} e i \textit{verb-noun} rappresentano l'esecuzione di azioni (\textit{verb}) su oggetti (\textit{noun}):
\begin{itemize}
	\item \textit{noun-verb}: si seleziona prima l'oggetto: si seleziona l'oggetto mentre esso stesso \`e nel luogo dell'attenzione; il luogo dell'attenzione si sposta sull'azione da compiere e a quel punto si esegue il gesto che attiva l'azione. Interrompere l'azione non richiede un'altra azione;
	\item \textit{verb-noun}: si seleziona prima l'azione (crea un modo).
\end{itemize}